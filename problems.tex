\documentclass[12pt]{article}
\usepackage{color,latexsym,setspace,graphicx,amsmath,amsfonts,amssymb,amsthm,enumerate,mathtools,adjustbox,ragged2e,hyperref}
\usepackage[margin=.7in]{geometry}
\renewcommand{\baselinestretch}{1.15}
\usepackage{tcolorbox}
\usepackage{color}
\usepackage{mdframed}
\usepackage[bitstream-charter]{mathdesign}
\usepackage[T1]{fontenc}
\mdfsetup{nobreak=true}

\newcommand{\bb}{\mathbb}

\newenvironment{problem}[1][]{\medskip \refstepcounter{section} \noindent  
\textbf{Problem \thesection.}\def\pnm{#1}\ifstrempty{#1}{}{\ (\textcolor{red}{#1})} \addcontentsline{toc}{subsection}{Problem \thesection.\ifstrempty{#1}{}{\ (\textcolor{red}{#1})}}}{ \smallskip \par}
\newenvironment{solution}[1][]{\smallskip \noindent {\textit{{Solution}.}\ }}{ \noindent \hfill $\square$  \smallskip \par}

\usepackage{tocloft}
\renewcommand{\cftdot}{.\ }


\begin{document}


\tableofcontents
\newpage

\begin{mdframed}
    \begin{problem}[Analysis]
        If for a function $f:\mathbb{R}\to \mathbb{R}$ image of each compact set is compact then $f$ is continous. T/F.
    \end{problem}
    \begin{solution}
        No, we can take the function $$f=\begin{cases}
            sin(\frac{1}{x})\ \text{if}\ x\neq 0,\\
            0\ \text{else}.\end{cases}$$
        This function is discontinous at 0.
    \end{solution}
\end{mdframed}

\begin{mdframed}
    \begin{problem}
        Existence of the limit $\lim_{n\to \infty} \frac{1}{1}+\frac{1}{2}+\dots+\frac{1}{n}-\text{log}n.$
    \end{problem}
    \begin{solution}
        Let $x_n= \frac{1}{1}+\frac{1}{2}+\dots+\frac{1}{n}-{log}n$. Then $x_{n+1}-x_n=\frac{1}{n+1}-log(\frac{n+1}{n}).$ But $log(1+x)\geq \frac{x}{x+1}$. Thus the sequence is decreasing and we can show(!) that it is bounded below.
    \end{solution}
\end{mdframed}


\begin{mdframed}
    \begin{problem}
        What is the smallest positive real numer $c$ such that $||x||_1\leq c||x||_{\infty}$ for all $x\in \mathbb{R}^n$.
    \end{problem}
    \begin{solution}
        Clearly $||x||_1\leq n||x||_\infty.$ Now, we claim that $c=n$. Let if possible $||x||_1\leq (n-\epsilon)||x||_\infty$ for some $\epsilon>0$, for all $x\in \mathbb{R}^n$. But for $x=(1,1,\dots,1)$ we will have $||x||_1=n, ||x||_\infty=1$ and hence $||x||>||x||_\infty.$
    \end{solution}
\end{mdframed}


\begin{mdframed}
    \begin{problem}
        If a group is finitely generated then there exist atmost finitely many subgroup of any index.
    \end{problem}
    \begin{solution}
        Let us consider $G$ be the group and $H$ be its subgroup such that $[G:H]=n.$ The group acts on the cosets $\{H,g_2H,\dots,g_nH\}=\{1,2,3,\dots,n\}$ and it induces a homomorphism \begin{center}
            $\varphi_H:G \to S_n$ such that $g\mapsto_{\varphi_H} \sigma_g$.
        \end{center} Now the stabilizer of the element $H$ in $G/H$ can be identified as $\{g\in G\mid \sigma_g=1\}$ i.e., $\{g\in G \mid gg_iH=g_iH, 1\leq i\leq n\}$ i.e., $H$. We claim that different subgroups $H$ and $H'$ will induce different maps. For $h\in H, h\notin H'$ we have $\varphi_H(h)=1$ but $\varphi_{H'}(h)\neq 1$. Again there are atmost finitely many maps from $G$ to $S_n$ and hence as a result there can exist only finite many subgroups of index $n$.
    \end{solution}
\end{mdframed} 

\begin{mdframed}
    \begin{problem}
        For primes $p>q>2$, group of order $pq^2$ contains a subgroup of ordre $pq$. 
    \end{problem}
    \begin{solution}
        The number of sylow $p$ subgroup $n_p$ divides $q^2$ as well as $p\mid n_p-1.$ Now $n_p$ is odd if it is equal to $q$ or $q^2$. Since $p$ is also an odd prime we can not have $p\mid n_p-1$ in this case. Thus we must have $n_p=1$ i.e., the sylow$-p$ subgroup, $H$ in $G$ is normal and has order $p$. Now by Cauchy's theorem there exists $b\in G$ of order $q$. Let $K=<b>$. Then $HK$ is the desired subgroup of $G$. 
    \end{solution}
\end{mdframed}

\begin{mdframed}
    \begin{problem}
        $SL_n$ is a product of matrices of the form $E_{ij}(a)=I+a\delta_{ij},1\leq i\neq j\leq n$.
    \end{problem}
    \begin{solution}
        Clearly $E_{ij}(a)\in SL_n$ and 
        \[\delta_{ij}\delta_{kl}=\begin{cases*}
            \delta_{il} &\ if $j=k$,\\
            0 &\ else.
        \end{cases*} \]
        implies \begin{align*}
            E_{ij}(a)E_{ij}(-a)&=(I+a\delta_{ij})(I-a\delta_{ij}) \\
            &=I-a^2\delta_{ij}\delta_{ij}\\
            &=I.
        \end{align*}
        For $A\in SL_n$, since not all entries in the first column can be zero we must have $a_{i1}\neq 0$ and $E_{1i}(1)A=(I+\delta_{1i})A=A+$ 
    \end{solution}
\end{mdframed}

\begin{mdframed}
    \begin{problem}
        $X$ be a compact metric space with atleast two points and $a\in X$. Then 
        \begin{enumerate}
            \item either $X\smallsetminus\{a\}$ is compact or $X$ is connected,
            \item but not both.
        \end{enumerate}
    \end{problem}
    \begin{solution}
        \begin{enumerate}
            \item Let us assume that $A=X\smallsetminus\{a\}$ is not compact then we know $A$ is not closed.

            \item Let us assume that $X$ is connected and if possible $X\smallsetminus\{a\}$ is compact. Then $X\smallsetminus\{a\}$ is closed. Also $\{a\}$ is a closed subset of $X$. This contradicts that $X=(X\smallsetminus\{a\})\cup\{a\}$ is connected. 
        
            Conversely if $A=X\smallsetminus\{a\}$ is compact then it will be closed in $X$ and we will have $X=A\cup B$, for $B=\{a\}$. Thus $X$ is not connected.
        \end{enumerate}
    \end{solution}
\end{mdframed}


\begin{mdframed}
    \begin{problem}
        $GL^{+}_n(\bb{R})$ and $GL^{-}_n(\bb{R})$ are homeomorphic.
    \end{problem}
    \begin{solution}
        We can define $\psi: GL^{+}_n(\bb{R})\to GL^{-}_n(\bb{R})$ such that $\psi(M)=AM$, where $A$ is a diagonal matrix such that $a_{11}=-1$ and $a_{ii}=1$ for $1<i\leq n.$ 
    \end{solution}
\end{mdframed}

\begin{mdframed}
    \begin{problem}
        Show that the General Linear group with positive determinant, $GL^{+}_n(\bb{R})$ is connected.
    \end{problem}
    \begin{solution}
        We know that $GL^{+}_n(\bb{R})=\det^{-1}((0,\infty))$ and hence it is open. If we can show that this there is some kind of homeomorphism we are through. 
    \end{solution}
\end{mdframed}

\begin{mdframed}
    \begin{problem}[Matrix, Topology]
        Show that $SL_2(\bb{R})$ is connected.
    \end{problem}
    \begin{solution}
        Here we will use the fact that the General Linear group with positive determinant, $GL^{+}_n(\bb{R})$ is path connected. With the help of this fact we can define a continous map \[\phi:GL^{+}_n(\bb{R})\to SL_n(\bb{R})\] such that \[\phi(A)=\frac{A}{(\det(A))^{\frac{1}{n}}}.\] Clearly this is a surjection and hence $SL_n(\bb{R})$ is connected.
    \end{solution}
\end{mdframed}

\begin{mdframed}
    \begin{problem}
        $f:\bb{R}\to \bb{R}$ is continous. Then show that $f$ is open iff it is strictly monotone.
    \end{problem}
    \begin{solution}
        Let us assume that $f$ is open and if possible there exist $a<b<c$ such that $f(a)<f(b)>f(c)$. Now if we restrict $f$ to the interval $[a,c]$, then its supremum, $M$ will exist and $M$ will strictly be greater than $f(a),f(c)$ i.e., $f([a,c])=[m,M]$. Therefore $f((a,c))$ will be a half closed interval i.e., either $f((a,c))=[m,M]$ or $f((a,c))=(m.M])$, contradicting our assumption that the map $f$ is open. 

        Conversely WLOG let us assume that $f$ is strictly increasing. It is sufficient to show that $f$ maps open intetrval to open sets. Now, $f$ being continous and strictly increasing implies $f((a,b))=(f(a),f(b)).$
    \end{solution}
\end{mdframed}

\end{document}