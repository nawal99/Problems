\documentclass[11pt]{article}
\usepackage{color,latexsym,setspace,graphicx,amsmath,amsfonts,amssymb,amsthm,enumerate,quiver,mathtools,adjustbox,ragged2e}
\usepackage[margin=1in]{geometry}
\usepackage[colorlinks,linkcolor=newgreen]{hyperref}
\usepackage{tcolorbox}
\usepackage{color}
\usepackage{mdframed}


\usepackage[bitstream-charter]{mathdesign}
\usepackage[T1]{fontenc}
\renewcommand{\baselinestretch}{1.2}

\mdfsetup{nobreak=true}

\definecolor{newgreen}{RGB}{0,150,0}
\definecolor{newtoc}{RGB}{0,51,102} %color definitions
\definecolor{newgray}{RGB}{64,64,64}
\definecolor{newpblm}{RGB}{0,102,0}

\newcommand{\bb}{\mathbb}
\newcommand{\sylow}{\textit{sylow}}
\newcommand{\spec}{\text{Spec}}
\newcommand{\Spec}{\text{Spec}}
\newcommand{\mf}{\mathfrak{m}}
\newcommand{\mfv}{\mathfrak{m}_v}
\newcommand{\ab}{\mathbb{A}}
\newcommand{\cbb}{\mathbb{C}}
\newcommand{\rbb}{\mathbb{R}}
\newcommand{\pf}{\mathfrak{p}}
\newcommand{\qf}{\mathfrak{q}}
\newcommand{\rf}{\mathfrak{r}}
\newcommand{\af}{\mathfrak{a}}
\newcommand{\qb}{\mathbb{Q}}
\newcommand{\subs}{\subseteq}
\newcommand{\sups}{\supseteq}
\newcommand{\oline}{\overline}
\newcommand{\pr}{A[x_1,x_2,\dots,x_n]}
\newcommand{\vphi}{\varphi}
\newcommand{\zb}{\mathbb{Z}}
\newcommand{\kn}{k[X_1,X_2,\dots,X_n]}
\newcommand{\km}{k[X_1,X_2,\dots,X_m]}
\newcommand{\xn}{X \subseteq \ab_k^n}
\newcommand{\ym}{X \subseteq \ab_k^m}
\newcommand{\empz}[1]{{\textcolor{red}{\emph{#1}}}}

\newenvironment{remark}[1][]{ \begin{mdframed} \textcolor{blue}{Remark.\ } }{\end{mdframed} \vspace{1.5cm} \par}
\newenvironment{problem}[1][]{\medskip \refstepcounter{section} \noindent  $\S$ 
{\textcolor{newgreen}{\scshape{Problem \thesection.}}}\def\pnm{#1}\ifstrempty{#1}{}{\ (\textcolor{red}{#1})} \addcontentsline{toc}{subsection}{\scshape{Problem} \thesection.\ }\itshape }{ \smallskip \par}

\newenvironment{solution}[1][]{\smallskip \noindent {\textit{{Solution}.}\ }}{ \noindent \hfill $\square$  \medskip \vspace{1.5cm} \par}

\usepackage{tocloft}
\renewcommand{\contentsname}{}
%\renewcommand{\cftdot}{}

\usepackage{titlesec,multicol}



\newcommand*{\setupTOC}{}

\begin{document}



\begin{center}
    \LARGE Collection of Problems\\
    \large Nawal Kishor Hazarika
\end{center}
\vspace{2cm}

\begin{center}
    \bfseries\Large \textcolor{newtoc}{Contents}
\end{center}
\begin{multicols}{2}
    {\setupTOC  \begingroup
    \hypersetup{linkcolor=newtoc}
    \tableofcontents
    \endgroup}
\end{multicols}
    
\newpage

    \begin{problem}[Analysis]
        If for a function $f:\mathbb{R}\to \mathbb{R}$ image of each compact set is compact then $f$ is continous. T/F.
    \end{problem}
    \begin{solution}
        No, we can take the function \[f=\begin{cases}
            \sin\left(\frac{1}{x}\right)\ \text{if}\ x\neq 0,\\
            0\ \text{else}.\end{cases}\]
        This function is discontinous at 0.
    \end{solution}



    \begin{problem}
        Existence of the limit $\lim_{n\to \infty} \frac{1}{1}+\frac{1}{2}+\dots+\frac{1}{n}-{\log}n.$
    \end{problem}
    \begin{solution}
        Let $x_n= \frac{1}{1}+\frac{1}{2}+\dots+\frac{1}{n}-{log}n$. Then $x_{n+1}-x_n=\frac{1}{n+1}-log(\frac{n+1}{n}).$ But $\log(1+x)\geq \frac{x}{x+1}$. Thus the sequence is decreasing and we can show(!) that it is bounded below.
    \end{solution}




    \begin{problem}
        What is the smallest positive real numer $c$ such that $||x||_1\leq c||x||_{\infty}$ for all $x\in \mathbb{R}^n$.
    \end{problem}
    \begin{solution}
        Clearly $||x||_1\leq n||x||_\infty.$ Now, we claim that $c=n$. Let if possible $||x||_1\leq (n-\epsilon)||x||_\infty$ for some $\epsilon>0$, for all $x\in \mathbb{R}^n$. But for $x=(1,1,\dots,1)$ we will have $||x||_1=n, ||x||_\infty=1$ and hence $||x||>||x||_\infty.$
    \end{solution}




    \begin{problem}
        If a group is finitely generated then show that there exist atmost finitely many subgroup of any given index.
    \end{problem}
    \begin{solution}
        Let us consider $G$ be the group and $H$ be its subgroup such that $[G:H]=n.$ The group acts on the cosets $\{H,g_2H,\dots,g_nH\}=\{1,2,3,\dots,n\}$ and it induces a homomorphism \begin{center}
            $\varphi_H:G \to S_n$ such that $g\xrightarrow{\varphi_H} \sigma_g$.
        \end{center} Now the stabilizer of the element $H$ in $G/H$ can be identified as $\{g\in G\mid \sigma_g=1\}$ i.e., $\{g\in G \mid gg_iH=g_iH, 1\leq i\leq n\}$ i.e., $H$. We claim that different subgroups $H$ and $H'$ will induce different maps. For $h\in H, h\notin H'$ we have $\varphi_H(h)=1$ but $\varphi_{H'}(h)\neq 1$. Again there are atmost finitely many maps from $G$ to $S_n$ and hence as a result there can exist only finite many subgroups of index $n$.
    \end{solution}
 


    \begin{problem}
        For primes $p>q>2$, group of order $pq^2$ contains a subgroup of ordre $pq$. 
    \end{problem}
    \begin{solution}
        The number of sylow $p$ subgroup $n_p$ divides $q^2$ as well as $p\mid n_p-1.$ Now $n_p$ is odd if it is equal to $q$ or $q^2$. Since $p$ is also an odd prime we can not have $p\mid n_p-1$ in this case. Thus we must have $n_p=1$ i.e., the sylow$-p$ subgroup, $H$ in $G$ is normal and has order $p$. Now by Cauchy's theorem there exists $b\in G$ of order $q$. Let $K=<b>$. Then $HK$ is the desired subgroup of $G$. 
    \end{solution}



    \begin{problem}
        $SL_n$ is a product of matrices of the form $E_{ij}(a)=I+a\delta_{ij},1\leq i\neq j\leq n$.
    \end{problem}
    \begin{solution}
        Clearly $E_{ij}(a)\in SL_n$ and 
        \[\delta_{ij}\delta_{kl}=\begin{cases*}
            \delta_{il} &\ if $j=k$,\\
            0 &\ else.
        \end{cases*} \]
        implies \begin{align*}
            E_{ij}(a)E_{ij}(-a)&=(I+a\delta_{ij})(I-a\delta_{ij}) \\
            &=I-a^2\delta_{ij}\delta_{ij}\\
            &=I.
        \end{align*}
        For $A\in SL_n$, since not all entries in the first column can be zero we must have $a_{i1}\neq 0$ and $E_{1i}(1)A=(I+\delta_{1i})A=A+$ 
    \end{solution}



    \begin{problem}
        $X$ be a compact metric space with atleast two points and $a\in X$. Then 
        \begin{enumerate}
            \item either $X\smallsetminus\{a\}$ is compact or $X$ is connected,
            \item but not both.
        \end{enumerate}
    \end{problem}
    \begin{solution}
        \begin{enumerate}
            \item Let us assume that $A=X\smallsetminus\{a\}$ is not compact then we know $A$ is not closed.

            \item Let us assume that $X$ is connected and if possible $X\smallsetminus\{a\}$ is compact. Then $X\smallsetminus\{a\}$ is closed. Also $\{a\}$ is a closed subset of $X$. This contradicts that $X=(X\smallsetminus\{a\})\cup\{a\}$ is connected. 
        
            Conversely if $A=X\smallsetminus\{a\}$ is compact then it will be closed in $X$ and we will have $X=A\cup B$, for $B=\{a\}$. Thus $X$ is not connected.
        \end{enumerate}
    \end{solution}




    \begin{problem}
        $GL^{+}_n(\bb{R})$ and $GL^{-}_n(\bb{R})$ are homeomorphic.
    \end{problem}
    \begin{solution}
        We can define $\psi: GL^{+}_n(\bb{R})\to GL^{-}_n(\bb{R})$ such that $\psi(M)=AM$, where $A$ is a diagonal matrix such that $a_{11}=-1$ and $a_{ii}=1$ for $1<i\leq n.$ 
    \end{solution}



    \begin{problem}
        Show that the General Linear group with positive determinant, $GL^{+}_n(\bb{R})$ is connected.
    \end{problem}
    \begin{solution}
        We know that $GL^{+}_n(\bb{R})=\det^{-1}((0,\infty))$ and hence it is open. If we can show that this there is some kind of homeomorphism we are through. 
    \end{solution}



    \begin{problem}[Matrix, Topology]
        Show that $SL_2(\bb{R})$ is connected.
    \end{problem}
    \begin{solution}
        Here we will use the fact that the General Linear group with positive determinant, $GL^{+}_n(\bb{R})$ is path connected. With the help of this fact we can define a continous map \[\phi:GL^{+}_n(\bb{R})\to SL_n(\bb{R})\] such that \[\phi(A)=\frac{A}{(\det(A))^{\frac{1}{n}}}.\] Clearly this is a surjection and hence $SL_n(\bb{R})$ is connected.
    \end{solution}



    \begin{problem}
        $f:\bb{R}\to \bb{R}$ is continous. Then show that $f$ is open iff it is strictly monotone.
    \end{problem}
    \begin{solution}
        Let us assume that $f$ is open and if possible there exist $a<b<c$ such that $f(a)<f(b)>f(c)$. Now if we restrict $f$ to the interval $[a,c]$, then its supremum, $M$ will exist and $M$ will strictly be greater than $f(a),f(c)$ i.e., $f([a,c])=[m,M]$. Therefore $f((a,c))$ will be a half closed interval i.e., either $f((a,c))=[m,M]$ or $f((a,c))=(m.M])$, contradicting our assumption that the map $f$ is open. 

        Conversely WLOG let us assume that $f$ is strictly increasing. It is sufficient to show that $f$ maps open intetrval to open sets. Now, $f$ being continous and strictly increasing implies $f((a,b))=(f(a),f(b)).$
    \end{solution}



    \begin{problem}[Group Theory, Sylow Theorems]
        What is the number of $\sylow-p$ subgroups in $GL_n(\bb{F}_p)$. 
    \end{problem}
    \begin{solution}
        We have $|G|=|GL_n(\bb{F}_p)|=(p^n-1)(p^n-p)\dots(p^n-p^{n-1})$. Therefore the cardinality of a $\sylow-p$ subgroup in $G$ is $p^{1+2+\cdots+(n-1)}=p^{\frac{(n-1)n}{2}}$. Now the subgroup $H$ of $G$ consisting of the upper triangular matrices with diagonal entries 1 is a $\sylow-p$ subgroup of $G$. Thus the number of $\sylow-p$ subgroup is same as the index of the normalizer of $H$ in $G$. We claim \[N=
        \left\{ A\in G\mid a_{ii}\neq 0, a_{ij}=0 \text{\ for \ }  i<j \right\}\] is equal to $N_H(G)$. $N\subs N_H(G)$ is obvious. 
        
        To proof the other direction we have to do some work. We have \[N = \left\{\begin{bmatrix}
            a_{11} & a_{12} & a_{13} & \dots & a_{1,n-1} & a_{1n}\\
            0 & a_{22} & a_{23}  & \dots &a_{2,n-1} & a_{2n}\\
            &&&\ddots\\
            0 & 0 & 0  & \dots &a_{n-1,n-1} & a_{n-1,n}\\
            0 & 0 & 0 & \dots & 0 & a_{nn}\\
            \end{bmatrix}
            \mid a_{ij} \in F_{p}, a_{ii} \ne 0 \right\}.
            \]
            Let us consider the subspace $V_i=\langle e_1,e_2,\dots,e_i \rangle$. It is clear that $HV_i\subs V_i.$ \textbf{First} we claim that this are the only subspaces such that $HU\subs U.$ If $u=(u_1,u_2,\dots,u_n)^t$ is some basis vector of $U$ with say $u_i\neq 0$. WLOG we can assume $u_i=1$. Now for $j\leq i$ \[ (I+\delta_{ji})u=(u_1,u_2,\dots,u_j+u_i,\dots,u_n)^t.\] Thus $(u_1,u_2,\dots,u_j+u_i,\dots,u_n)-(u_1,u_2,\dots,u_j,\dots,u_n)=(0,0,\dots,u_i,\dots,0)=e_j$ is contained in $U$. Therefore we can conclude that $U=V_j$, where $j$ is largest index such that a basis vector has a nonzero $j$th entry.

            Now for any $g\in N_G(H)$ and $h\in H$, $ghg^{-1}\in H.$ Therefore $gh=h'g$ for some $h'\in H.$ Again we claim $hV_i=V_i$ foe each $i$. Since $he_i=(h_{1i},h_{2i},\dots,h_{ni})^t$, $he_1=(h_{1i},0,\dots,0)=e_1$. Again $he_2=(h_{12},1,\dots,0)^t=h_{12}e_1+e_2$ i.e., \[he_2-h(h_{12}e_1)=e_2\in hV_i.\] By this way we have $hV_i=V_i$. Therefore $ghV_i=gV_i=h(gV_i)$ i.e. $h(gV_i)\subs gV_i$ and $H(gV_i)\subs gV_i.$ From our first claim we have $gV_i=V_j$ for some $1\leq j\leq n$. Since $g$ is invertible and it preserves rank we must have $gV_i=V_i$ for each $1\leq i \leq n$. Thus we have $g\in N$ by simple observation.
    \end{solution}

    \begin{problem}[Compelx Analysis]
        Find the entire functions $f:\mathbb{C}\to \mathbb{R}.$
    \end{problem}
    \begin{solution}
        If such an entire function $f(z)$ exists then the function $if(z)$ is also entire and so is $\exp^{if(z)}$. This gives us $\mid \exp^{if(z)} \mid=1$ and by Liouville's theorem it is constant. Consequently $f(z)$ must be constant.
    \end{solution}

\begin{problem}
    A subgroup $H$ of index 5 in an odd order group $G$ is normal. 
\end{problem}
\begin{solution}
    Since $|G:H|=5$, we get a homomorphism $\vphi : G\to S_5$ and $K=\ker(\vphi)\subs H.$ Thus $|G:K|\geq 5$. The subgroups of odd order in $S_5$ can have order $3,5$ or $15$. Now $|G:K|=5$ implies $H=K$ and hence $H\trianglelefteq G.$ Otherwise $G/K$ is a group of order 15. We know that any subgroup of $S_n$ either contains all even permutations or exactly half of them. If there are exactly half of the elements in $G/K\cong P\subs S_5$ are even permutations then $\sigma:P\to \{1,-1\}$ is a surjection. This gives us $|P:\ker(\sigma)|=2$ i.e., $2\mid |P|$, which is a clear contradiction to the fact that $|G|$ is odd. Thus all the elements in $P$ are even permutations i.e., $G/K\subs A_5$. But $A_5$ has no subgroup of order 15, another contradiction.
\end{solution}

\begin{problem}
    Let $G$ be a finite group and $H$ a subgroup of $G$ of prime index $p$. If $\gcd(|G|,p-1)=1$ then $G'\subs H$.
\end{problem}
\begin{solution}
    To be contd...
\end{solution}

\begin{problem}[Group Theory]
    A finite simple group $G$ does not have a normal subgroup of index $n$ if $|G|$ does not divide $N!$.
\end{problem}
\begin{solution}
    Let $|G:H|=n$ then we get a homomorphism $\vphi:G\to S_n$ induced by the action of $G$ on the cosets of $H$. Now $K=\ker(\vphi)\subs H$ and is a normal subgroup of $G$. $G$ being a simple group implies that $K=1$ i.e., $G$ is embedded in $S_n$. Thus $G$ can be thought as a subgroup of $S_n$ and hence $|G|\mid n!.$ \\
    \emph{Consequences-} $A_5$ has no subgroup of order $15$ and $20$ since $15\nmid 24$ and $20\nmid 6.$
\end{solution}

\begin{problem}[Real Analysis, Continous Functions]
    Periodic continous function $f:\mathbb{R} \to \mathbb{R}$ is uniformly continous.
\end{problem}
\begin{solution}
    For ease of calculation we will consider the period of $f$ to be $1$. Now $f$ is continous on $[0,2]$ and for given $\epsilon>0$ there exists $\delta>0$ such that whenever $x,y\in [0,2]$ with \begin{center}
        $|x-y|<\delta$ implies $|f(x)-f(y)|<\epsilon.$
    \end{center} For any $x,y \in \mathbb{R}$ with $x>y(>0,\ \text{say})$ there exist $n,m\in \mathbb{N}$ such that $x=n+r,y=m+s$ with $0\leq r,s<1$. For $\delta<1$ if $|x-y|<\delta$ we claim that $n=m$ or $n=m+1.$ Otherwise let if possible $n\geq m+2$ then {\allowdisplaybreaks \begin{align*}
        x-y
        &= (n+r)-(m-s)\\
        &\geq 2+(r-s).    
    \end{align*}}
     But  $0\leq r<1$ and $0\leq s<1$ give us $-1\leq r-s\leq 1$ i.e., $x-y\geq 2-1=1.$ This contradicts $|x-y|<\delta<1$. Therefore if we choose $\delta'=\min\{\delta,1\}$ then $|f(x)-f(y)|<\epsilon$ whenever $|x-y|\leq \delta'.$
\end{solution}

\begin{problem}[Topology, Metric Space]
    The complement of a proper subspace $W$ of $\mathbb{R}^n$ is connected if and only if $\dim(W)\leq n-2$.
\end{problem}
\begin{solution}
    Let us consider a proper subspace $W$ such that $\dim(W)>n-2$, therefore $\dim(W)=n-1$ and $W^{\perp}$ is of dimension 1. If $W^{\perp}=\text{span}\{v\}$ then we can consider the continous function $g:\mathbb{R}^n\to\mathbb{R}$ defined by $f(x)=\langle v,x \rangle$. In this case $f^{-1}(0)=W$ and $f^{-1}(\mathbb{R}\smallsetminus\{0\})=\mathbb{R}^n\smallsetminus W$. Thus we obtain two open sets $A=f^{-1}((0,\infty))$ and $B=f^{-1}((-\infty,0))$ such that $A\cup B=\mathbb{R}^n\smallsetminus W$ i.e., the complement $W$ is not connected. Hence for the complement of a proper subspace of $\mathbb{R}^n$ to be connected we must have $\dim(W)\leq n-2.$ 

    Conversely let us assume that $\dim(W)\leq n-2$. We need to show that $\mathbb{R}^n\smallsetminus W$ is connected. The idea is to project any two vectors $x,y\in \mathbb{R}^n\smallsetminus W$ to $W^{\perp}$, which is path connected. By this we get the path $x\to x'\to y'\to y.$ \[\begin{tikzcd}
        &&& y \\
        x \\
        &&& {y'} \\
        {x'}
        \arrow[from=2-1, to=4-1]
        \arrow[curve={height=18pt}, from=4-1, to=3-4]
        \arrow[from=3-4, to=1-4]
    \end{tikzcd}\]
    Let $\{e_1,e_2,\dots,e_k\}$ is an orthonormal basis for $W$ and $\{e_{k+1},\dots,e_n\}$is an orthonormal basis for $W^{\perp}$. The projection $x'$ of a vector $x=\sum_{i=1}^{n}x_ie_i$ onto $W^{\perp}$ is given by $\sum_{i=k+1}^{n}\langle x,e_i \rangle e_i=\sum_{i=k+1}^{n}x_ie_i.$ We claim that the straight line connecting $x$ and $x'$ lies on $W^c$.
\end{solution}

\begin{problem}[Complex Analysis]
    Entire function $f:\mathbb{C}\to \mathbb{C}$ with $\Im(f)>0$ is constant.
\end{problem}
\begin{solution}
    For an entire function $f$, $\exp^{-if(z)}$ is also an entire function and $|\exp^{-if(z)}|=|\exp^{\Im(f)}|$. Similarly $\exp^{if(z)}$ is entire and $|\exp^{if(z)}|=|\exp^{-\Im(f)}|<1.$ Therefore $\exp^{if(z)}$ is constant and so is $f(z)$
\end{solution}

\newpage
\begin{problem}[Functional Analysis]
    Let $X,Y,Z$ are Banach spaces such that $A:X\to Y$ and $B:Y\to Z$ are linear maps. If $BA, B$ are bounded and $B$ is injective then $A$ is also bounded. 
\end{problem}
\begin{solution}
    Let $x_n\to x$ and $A(x_n)\to y$. $B$ being bounded implies $B(A(x_n))\to B(y)$. Moreover $(BA)(x_n)\to (BA)(x)$ and $B$ is injective. Therefore $BA(x)=B(y)$ implies $A(x)=y$ and hence $A$ is a closed map. Hence $A$ is a bounded linear operator.
\end{solution}


\begin{problem}
    Evaluate the limit \[\pi \lim_{n\to \infty}\frac{\sum_{k=1}^{n} \sin\left(k\frac{\pi}{n}\right)}{n}.\]
\end{problem}
\begin{solution}
    We know that for an integrable function $f:[a,b]\to \mathbb{R}$ \[\int_{a}^{b}f(x)dx= \lim_{n\to \infty} \sum_{k=1}^{n}f(x_k)\Delta x\] where $x_k=a+k\Delta x$ and $\Delta x=\frac{b-a}{n}$. Comparing with the given function with the standard result we get $a=0,\frac{b-a}{n}=\frac{\pi}{n}$ i.e., $b=\pi.$ Thus \begin{align*}
        \pi \lim_{n\to \infty}\frac{\sum_{k=1}^{n} \sin\left(k\frac{\pi}{n}\right)}{n}
        &=\lim_{n\to \infty}{\sum_{k=1}^{n} \sin\left(k\frac{\pi}{n}\right)}\frac{\pi}{n}\\
        &= \int_{0}^{\pi}\sin\left( x\right)dx\\
        &= {[-\cos x]^{\pi}}_{0}\\
        &= 2.
    \end{align*}
\end{solution}

\begin{problem}[Linear Algebra, Topology]
    The set of rank two matrices in $M_{2\times 3}$ is open.
\end{problem}
\begin{solution}
    The required set is the inverse image of $\mathbb{R}^3\setminus (0,0,0)$, where $f:M_{2\times 3}\to \mathbb{R}$  is a continous map given by $f(A)=f(A_1,A_2,A_3)=(\det(A_1,A_2),\det(A_2,A_3),\det(A_3,A_1))$. Inverse image of $(0,0,0)$ is the set of all matrices of rank less than or equal to 1. Each $\det(A_i,A_j)$ map is continous because they are polynomials in the entries of $A$. Consequently by \textit{mapping into products}  the map $f$ is continous.
\end{solution}

\begin{problem}[Linear Algebra, Topology]
    The orthogonal matrices of size $n\times n$  over $\mathbb{R}$, $\mathcal{O}_n(\mathbb{R})$ is compact. Is $\mathcal{O}_n(\cbb)$ compact?
\end{problem}
\begin{solution}
    For any $A\in \mathcal{O}_n(\mathbb{R})$ we have $AA^T=I_n$. Now $(AA^T)_{ij}=\sum_{j=1}^{n}{a_{ij}}^2$ i.e., for each $i,j$ the term $|a_{ij}|\leq 1$. Thus the elements of the set are bounded above by $n^2$, since \[||A||=\sqrt{\sum_{i=1}^n\sum_{j=1}^{n}{a_{ij}}^2}\leq n^2.\]
    To show that the given set is infact a closed set in $M_n(\mathbb{R})$ we consider the map $f:M_n(\rbb) \to M_n(\rbb)$ given by $f(A)=AA^T.$ We claim that this map is continous. For a sequence $A_n\to A$ we must have ${a_{ij}}^{(n)}\to a_{ij}$ for each $i,j.$ Therfore \[{f(A_n)=\left(A^{(n)}\left(A^{(n)}\right)^T\right)_{ij}}=\sum_{j=1}{a_{ij}}^{(n)}{a_{ji}}^{(n)}\to \sum_{j=1} a_{ij}a_{ji}=(AA^T)_{ij}.\] Thus $f$ is continous and the inverse image of the closed set $\{I_n\}$(\textit{singleton set in a metric space is closed}) is precisely $\mathcal{O}_n(\rbb)$.
    
    To show that $\mathcal{O}_2(\cbb)$ is not compact we need to find an unbounded matrix \[A=\begin{bmatrix}
        a & b\\
        c& d
    \end{bmatrix}\] such that $a^2+b^2=c^2+d^2=1, ac+bd=0$. We can consider $a=i\sqrt{n}=d,b=-c=\sqrt{n+1}.$ In this case we get unbounded matrices in $\mathcal{O}_n(\cbb)$ for $n=1,2,3,\dots$ because \begin{align*}
        AA^T
        &=\begin{bmatrix}
            i\sqrt{n} & \sqrt{n+1}\\
            -\sqrt{n+1} & i\sqrt{n}
        \end{bmatrix}
        \begin{bmatrix}
            i\sqrt{n} & -\sqrt{n+1}\\
            \sqrt{n+1} & i\sqrt{n}
        \end{bmatrix}\\
        &=\begin{bmatrix}
            -n+(n+1) & i\sqrt{n^2+n}-i\sqrt{n^2+n}\\
            -i\sqrt{n^2+n}+i\sqrt{n^2+n} & (n+1)-n
        \end{bmatrix}\\
        &=I_n.
    \end{align*} 
\end{solution}

\begin{problem}
    For a finite group $G$ of order $n$ with a subgroup $H$ of order $m$, $\left(\frac{m}{n}\right)!|<2n$ implies $G$ is not simple.
\end{problem}
\begin{solution}
    Let us consider $\vphi:G\to S_{\frac{n}{m}}$ induced by the action of $G$ on the cosets of $H$. Now $K=\ker(\vphi)\trianglelefteq G$. If $K\neq\{1\}$ then $G$ is not simple and we are done. Otherwise $G$ is isomorphic to a subgroup of $S_N,N=\frac{n}{m}.$ By Lagrange's theorem $n\mid N!$, but $N!<2n$ implies $N!=n$ i.e., $G\cong S_N$. Therfore $G$ is not simple as $S_N$ has a normal subgroup $A_N$ for each $N\in \mathbb{N}$.
\end{solution}

\begin{problem}[Metric Spaces]
    Let $X,Y$ be topological spaces such that $Y$ is normal. Fruthermore for the function $f:X\to Y$ and for every continous function $\vphi:Y\to \rbb$, $\vphi\circ f$ is continous. Prove that $f$ is continous.
\end{problem}
\begin{solution}
    Let us consider a closed set $C$ in $Y.$ For a point $p\notin f^{-1}(C)$ we consider the two closed sets $\{f(p)\}$ and $C$. By normality of $Y$ and Uryshon's Lemma there exists a continous function $\vphi:Y\to \rbb$ such that $\vphi(f(p))=0$ and $\vphi(c)=0$ for all $c\in C.$ We define $g=\vphi\circ f$. Then $g(p)=0$ and for any $x\in f^{-1}(C)$ the image of $x$ under $g$ is $g(x)=\vphi(f(p))=1$ i.e., $g(f^{-1}(C))=\{1\}.$ Since $Y$ is normal, there exist two disjoint open sets $U$ and $V$ in $\rbb$ such that $\{0\}\subs U$ and $\{1\}\subs V$. Given that $g$ is continous. Hence $U'=g^{-1}(U)$ and $V'=g^{-1}(V)$ are two disjoint open sets in $X$. Clearly $p\in U'$ as $g(p)=0\in U$ and $f^{-1}(C)\subs V'$. Therefore we get a open neighbourhood $U'$ of $p$ such that $U'\cap f^{-1}(C)=\emptyset$ i.e., $\subs X\smallsetminus f^{-1}(C) $. Hence $X\smallsetminus f^{-1}(C)$ is open and the set $f^{-1}(C)$ is closed.

    \begin{center}
        \begin{tikzcd}
            \draw (2,0) ellipse (2cm and 3cm);
            \draw (2,.6) ellipse (.7cm and 1.3 cm) node {f^{-1}(C)};
            \filldraw[red] (2,-2) circle(1pt) node[anchor=west]{p};
            \draw[dashed] (2,-2) circle (.4cm);
            \filldraw (2,-1.5) node {U'};
            \filldraw (2,2.3) node {V'};
            \draw[dashed] (2,.6) ellipse (1cm and 1.6 cm);
    
            \draw (7,0) ellipse (2cm and 2.5cm);
            \draw (7,1) ellipse (1cm and 1.2cm) node {C};
    
    
            \draw (12,0) ellipse (2cm and 3cm);
            \draw[dashed] (12,1) circle (.7cm);
            \draw[dashed] (12,-1) circle (.7cm);  
            \filldraw[black] (12,1.8) node {V};
            \filldraw[black] (12,-.2) node {U};
            \filldraw[blue] (12,1) circle(1pt) node [anchor=south]{1};
            \filldraw[blue] (12,-1) circle(1pt) node [anchor=south]{0};
    
            \draw[dashed] (2,-2) .. controls (5.5,-4) and (7,-4) .. (12,-1);
            \draw (6,-4) node {g=\vphi\circ f};
        \end{tikzcd}
    \end{center}
\end{solution}

\begin{problem}[Metric Spaces]
    Let $X,Y$ and $Z$ are metric spaces, $f:X\to Y$ is a continous onto map and $g:Y\to Z$ is such that $g\circ f$ is continous. If $X$ is compact prove that $g$ is also a continous map.
\end{problem}
\begin{solution}
    Let us consider a closed set $C$ in $Z$. Now $(g\circ f)^{-1}(C)$ is closed in $X$ and hence compact. $f$ being a continous map implies $f((g\circ f)^{-1}(C))$ is compact in $Y$ and hence a closed subset of $Y$. 
    
    We claim that $g^{-1}(C)=f((g\circ f)^{-1}(C))$. For any $y\in g^{-1}(C)\subs Y$ there exists $x$ in $X$ such that $f(x)=y$ because $f$ is onto. Now $g(y)\in C$ i.e., $g(f(x))\in C$. Thus $x\in (g\circ f)^{-1}(C)$ and hence $y=f(x) \in f((g\circ f)^{-1}(C)).$ Conversely for $w\in f((g\circ f)^{-1}(C))$, there exists $u\in (g\circ f)^{-1}(C)$ such that $f(u)=w.$ Now $(g\circ f)(u)\in C$ implies $f(u)\in g^{-1}(C)$ i.e., $w\in g^{-1}(C)$
\end{solution}

\begin{problem}
    Let $\{a_i:i\in \rbb\}$ is a set of non negative real numbers in $\rbb.$ If $\sup\{\sum_{i\in F}a_i \mid F\subs \rbb, |F|<\infty \}$ is finite. Show that except for countably many $a_i$'s rest all are zero. Also show that the the 'countably' can not be replaces by 'finite'. 
\end{problem}
\begin{solution}
    Let $F_n=\left\{i\in \rbb\mid a_i\geq \frac{1}{n}\right\}$ and $F_0=\cup_{n} F_n=\{i \in \rbb \mid a_i\neq 0\}.$ Now each of the $F_n$ must be finite, Otherwise for some $N_0$ and each $N\in \mathbb{N}$ \[\sup\left\{\sum_{i\in F}a_i \mid F\subs \rbb, |F|<\infty \right\} \geq \sup\left\{\sum_{i\in F}a_i \mid F\subs F_{N_0}, |F|<\infty \right\}\geq \frac{N}{N_0 }.\] As $N\to \infty$ the supremum becomes unbounded.

    $a_i=\frac{1}{i^2}$ for $i\in \mathbb{N}$ and zero at other points satisfies the above condition.
\end{solution}

\begin{problem}
    If $f:\{z\in \cbb\mid |z|>1\}$ is defined by $f(z)=\frac{1}{z}$, show that there does not exist any entire function $g$ such that $g=f$ on $|z|>1.$
\end{problem}
\begin{proof}
    If such a function exists, for $|z|>1$ it will be bounded above by $1$. Also on the comapct set $|z|\leq 1$ it will again be bounded. Thus $g$ is an entire bounded function and hence constant by Liouville's theorem. This a contradiction to the assumption that $g(z)=\frac{1}{z}$ for $|z|>1.$
\end{proof}


\begin{problem}
    Any entire function $f$ is either a polynomial or it has an essential singularity at $\infty$.
\end{problem}
\begin{solution}
    Let us conisder the taylor series expansion of $f$ about $0$. If it terminated after finite terms we are done. Otherwise $g(z)=f\left\{ \frac{1}{z}\right\}$ will have an essential singularity at zero i.e., $f$ will have an essential singularity at $\infty.$
\end{solution}

\begin{remark}
    All non-constant functions that are analytic everywhere in the complex plane, $\cbb$ must be unbounded at $\infty$ and hence have a singularity at $\infty.$
\end{remark}

\begin{problem}
    Does there exist a continous surjection from $[0,1)$ onto $\rbb?$
\end{problem}
\begin{solution}
    Yes, $f(x)=x\sin x.$
\end{solution}

\begin{problem}[Real Analysis, Integration]
    $f:\rbb \to \rbb$ is such that $\int_{-\infty}^{\infty}f<\infty$. Then show that the function $F(x)=\int_{-\infty}^{x}f(t)dt$ is uniformly continous.
\end{problem}
\begin{solution}
    Let us conisder $\int_{-\infty}^{\infty}f(x)dx=M<\infty$. For any $x>a$, \begin{align*}
        |F(x)-F(a)|
        &= \left| \int_{-\infty}^{x}f(t)dt -\int_{-\infty}^{a}f(t)dt \right|\\
        &=\left| \int_{a}^{x}f(t)dt \right|\\
        &\leq \int_{a}^{x}|f(t)|dt.
    \end{align*}
    Now $M=sup\{|f(t)| \mid {t\in[a,x]} \}$ is bounded becuase $wefwm$
\end{solution}

\begin{problem}[Topology, Compactness]
    Let $X$ be a topological space and $f:X\to[0,1]$ is a closed continous surjection and for each $a\in [0,1]$, $f^{-1}(a)$ is compact in $X$. Prove or disprove that $X$ is compact.
\end{problem}
\begin{solution}
    For any $a\in [0,1]$ and any nbd $U$ containing $f^{-1}(a)$ we claim that there exists an open nbd $W$ of $a$ such that $f^{-1}(W)\subs U$. Because $f(X\smallsetminus U)$ is closed in $[0,1]$ and hence $W=[0,1]\setminus f(X\setminus U)$ is a open set. For any $y\in f^{-1}(W)$, $f(y)\in [0,1]\setminus f(X\setminus U)$. Thus $f(y)\notin f(X\setminus U)$ and hence $y\notin X\setminus U$. This implies $y\in U.$

    Now $\{U_i\}$ be an open cover for $X$. For each $a\in [0,1]$ there exists an open set $U_a$ from the open cover such that $f^{-1}(a)\in U_a$. Thus we obtain a open nbd $W_a$ of $a$ such that \empz{$f^{-1}(W_a)\subs U_a.$} The collection of open sets $\{W_a\mid a\in [0,1]\}$ forms an open cover for $[0,1]$. Since $[0,1]$ is compact we have \empz{$\bigcup_{i=1}^{n} W_{a_i}=[0,1]$}. But $f$ is a surjection implies $f^{-1}\left(\bigcup_{i=1}^{n} W_{a_i}\right)=X.$ Thus \empz{$\bigcup_{i=1}^{n} f^{-1}(W_{a_i})=X$} and hence $\bigcup_{i=1}^{n} U_{a_i}=X.$  
\end{solution}

\begin{problem}[Topology, Normal Spaces]
    Let $f:X\to Y$ is a closed, continous, surjective map between two topological spaces. If $X$ is normal prove that $Y$ is also normal.
\end{problem}
\begin{solution}
    For any two disjoint closed sets $C,D$ in $Y$, the sets $A=f^{-1}(C),B=f^{-1}(D)$ are also closed in $X$. Moreover they are disjoint becuase $x\in f^{-1}(C)\cap f^{-1}(D)$ implies $f(x)\in C\cap D.$ $X$ being normal gives us two disjoint open sets $U,V$ such that $A\subs U, B\subs V$. Now the sets $X\setminus U,X\setminus V$ are closed in $X$ and so are $U'=f(X\setminus U)$ and $V'=f(X\setminus V)$ in $Y$ because $f$ is a closed map.
    
    We claim that $Y\setminus U',Y\setminus V'$ are disjoint open sets and $C\subs Y\setminus U', D\subs Y\setminus V'.$ The fact that they are open is straight foreward. Now \[(Y\setminus U')\cap (Y\setminus V')= Y\setminus(U'\cup V') = Y \setminus (f(X\setminus U) \cup f(X\setminus V)).\]
    But for any $f(x)=y\in Y$, $x\in U\cap V$ since $U\cap V=\phi.$ i.e., $x\in X\setminus U$ or $x\in X\setminus V.$ Thus $f(x)=y\in f(X\setminus U)\cup f(X\setminus V)$, which means $Y=f(X\setminus U)\cup f(X\setminus V).$ Therefore the sets $Y\setminus U'$ and $Y\setminus V'$ are disjoint open sets.

    Our proof will be complete if we can show that $C\subs Y\setminus U'$ and $D\subs Y\setminus V'.$ For $c\in C,$ there exists $a\in A=f^{-1}(C)$ such that $f(a)=c\in C$ i.e., $a\in A\subs U$. If $c\notin Y\setminus f(X\setminus U)$, $c\in f(X\setminus U)$. There will be some $a'\in X\setminus U$ such that $f(a')=c.$ This is a contradiction since $f^{-1}(C)\subs U.$
\end{solution}

\begin{problem}[Group Theory]
    Let $G$ be a group with the property that for some $a\in G$, $H=G\setminus \{a\}$ is a subgroup of $G$. Prove that $|G|=2.$
\end{problem}
\begin{solution}
    For any $b\in H$, $ab\notin H$. Otherwise $a=(ab)b^{-1}\in H.$ Thus for each $b\in H,$ $ab=a$ i.e., $b=1$. Since $H$ is a subgroup $a\neq 1.$ Thus there are only two elements in $G$, namely $1$ and $a.$ 
\end{solution}

\end{document}